\documentclass[10pt, onecolumn, draftclsnofoot, letterpaper, compsoc]{IEEEtran}

\usepackage{cite}
\usepackage{hyperref}
%usepackage{enumitem}
\usepackage{graphicx}

\graphicspath{ {images/} }

%%%%%%%%%%%%%%%%%%%%%%%%%%%%%%%%%%%%%%%%%%%%%%%%%%%%%
% Macro for the signatures at the end                %
\newcommand*{\SignatureAndDate}[1]{%
    \par\noindent\makebox[2.5in]{\hrulefill} \hfill\makebox[2.0in]{\hrulefill}%
    \par\noindent\makebox[2.5in][l]{#1}      \hfill\makebox[2.0in][l]{Date}%
}%

\renewcommand*\contentsname{Table of Contents} % Rename ToC

\newcommand{\myindent}{\hspace{\oldparindent}}

\usepackage{cite}

% Temp title and author
\title{Requirements}
\date{\today} % Somehow this isn't working..
\author{Totality AweSun \\
		Bret~Lorimore, Jacob~Fenger, George~Harder \\
		\textit{November 4th, 2016 \\
		CS 461 - Fall 2016}}

\begin{document}

%\setlist[itemize]{topsep=1pt} % EDIT LISTS

\maketitle

% George's Section
\section{Eclipse Image Processor}

The eclipse image processor is the piece of our project that handles the spatial
and temporal alignment of the eclipse images that are uploaded to the Eclipse
Megamovie website. We have identified three pieces\cite{imgKrista} into which
the image processor can be broken down. These are: image classification and
manipulation, the runtime environment, and circle detection. In order for this
element of our project to operate effectively it is critical that these three
pieces utilize robust and functional technologies. This section of the
technology review details what options are available for implementing each of
these three pieces, analyzes these options, and arrives at a determination as to
which option is the best in the context of this project.

% Bret's Section
\section{Image Processor Manager}

% Jake's Section
\section{Eclipse Simulator}
The eclipse simulator will be a standalone JavaScript module enabling users to
“preview” the eclipse. It will be designed in a stylized, 2D manner.
The simulator will incorporate a time slider to allow users to simulate
the eclipse in a time window spanning from 12 hours before the eclipse to 12
hours after it. As users drag the time slider, the eclipse will animate in the
simulator window. The view of the eclipse which users are presented will be 
specific to a location that the user enters. Additionally, the time will be
displayed in the simulator based on what location the user enters and the
positioning of the time slider.

\bibliographystyle{IEEEtran}
\bibliography{tech}

\end{document}
