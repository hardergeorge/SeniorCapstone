\documentclass[10pt, onecolumn, draftclsnofoot, letterpaper, compsoc]{IEEEtran}

\usepackage{graphicx}
\usepackage{amssymb}
\usepackage{amsmath}
\usepackage{amsthm}
\usepackage{alltt}
\usepackage{color}
\usepackage{url}
\usepackage{minted}

\graphicspath{ {images/} }

\renewcommand*\contentsname{Table of Contents} % Rename ToC

% Temp title and author
\title{Midterm Progress Report}
\author{Totality AweSun \\
		Bret~Lorimore, Jacob~Fenger, George~Harder \\
		\textit{May 15, 2017 \\
		CS 463 - Spring 2017}}

\begin{document}

\maketitle

\begin{abstract}
This document describes the current state of the \textit{North American Solar Eclipse 2017}
senior capstone project. The document gives a brief overview of the project and its components,
describes the current state of the project, describes problems that have been
encountered throughout the term, shows some of the code that has been produced thus far, gives
a week-by-week outline of progress throughout the term, and reflects over the term in the
retrospectives section at the end.
\end{abstract}

\newpage

\tableofcontents

\newpage

%%%%%%%%%%%%%%%%%%%%%%%%
%   Project Overview   %
%%%%%%%%%%%%%%%%%%%%%%%%
\section{Project Overview}

The North American Solar Eclipse 2017 Senior Capstone project is partnered
with Google to build a set of applications that will assist the development of
the Eclipse Megamovie Project. The overall project has been broken down into
three components: the eclipse image processor, the image processor manager, and
the solar eclipse simulator. Each will be individually outlined in the sections
below. \\

\subsection{Image Processor}

The image processor’s primary activity is to quickly and consistently identify
images of an eclipse at totality. The Eclipse Megamovie project will be
collecting thousands of images from photographers around the country, and the
image processor needs to identify the images of the eclipse at totality so that
these can then be stitched into a timelapse movie. In order to make the
stitching as easy as possible for the Eclipse Megamovie team, the image
processor will add metadata to each processed image that includes spatial
information about where the image was taken along the path of totality and
temporal information about how far into totality the eclipse is.


The purpose of the Image Processor is not to process many images as quickly as
possible. Instead, our goal is to be able to consistently and accurately process
a single image at a time. As such, the image processor will fit into the larger
project as an executable file that is called by the Image Processor Manager.
This allows us to focus the image processor solely on a single goal, and leave
parallelization and deployment to a different piece of the project. \\

\subsection{Image Processor Developer Pipeline}

The image processor developmer pipeline is a tool that will assist any future
development of the image processor. It consists of a script that operates
several primary stages that provides developers with an easily manipulatable
interface to the image processor. The script first downloads images from a
Google Cloud Storage (GCS) bucket. Then it runs the images through the image
processor in one of two modes: "batch" or "window." The first mode runs through
all of the images at once and the second lets the user see the results of a run
one image at a time. After the image processor finishes running and writing its
output, the script then uploads all of the processed images and the output into
a GCS bucket so the information can be easily found and reviewed. Lastly, the
pipeline calls a python script that formats the output of the image processor
into an HTML document. The HTML document contains metadata about the run of the
image processor so that runs can be recreated and easily compared. In addition,
it has a table with information about each individual image processed in the run.

The image processor developer pipeline is a valuable tool because it gives
developers very fine control over runs of the image processor. The script that
runs the pipeline takes arguments that control the mode, where to download
images from, where to upload the to, and what paramters to pass to the image
processor. This allows developers to easily experiment with small and large
changes to the image processor and examine results. \\

\subsection{Eclipse Simulator}

The eclipse simulator is an independent JavaScript module that has been
added to the existing Eclipse Megamovie webpage. This simulator will allow
users to "preview" a 2D stylized depiction of the eclipse from a given location
in the United States. Users interact with the simulator using a time slider
that ranges from 1.5 hours before to 1.5 hours after the eclipse, a "play" button,
location selection via a map or search bar, and a zoom mode.

To help with the eclipse ephemeris computations, we used an external
JavaScript library called MeeusJs. For the front end view for the simulator,
we utilized HTML5 SVG. Our eclipse simulator uses a model-view-controller 
architecture for controlling the states of each component as well as handling
the interactions. This architecture was chosen due to the ability to easily
exchange a component without altering the whole design of the system. For
example, if one wanted to create a whole new front end for the simulator,
they would not need to rewrite the model or controller component of the system.
They would simply need to ensure that the new view component can handle the
interactions with the controller module. \\


%%%%%%%%%%%%%%%%%%%%%%%%
%   Current Status     %
%%%%%%%%%%%%%%%%%%%%%%%%
\section{Current Status of the Project}

Based on the requirments we set for ourselves and agreed upon with our sponsor
our project is effectively complete. We have delivered a solar eclipse simulator
that meets the specifications our sponsor requested. In addition, we have
produced an image processing development pipline that will assist in future
development of our image processor. \\

\subsection{Image Processor}

As of the code freeze on May 1, we had a working image processor that identified
circles in images of eclipses and drew the two most prominent circles
(ostensibly the sun and moon) over the image. This tool meets ours and our
sponsor's specifications for the image processor. In addition, during meetings
with our sponsor he has expressed the fact that he is happy with our work and
appreciates what we have done. \\

\subsection{Image Processor Developer Pipeline}

The image processor developer pipeline meets the specifications our client
requested for a platform that will aid in future development of the image
processor. The developer pipeline is a tool that allows developers to quickly
and easily run and experiment with different paramaters or tweaks to code. This
tool gives our sponsor, who plans to continue development on the image processor,
an effective way to validate his development is improving the image processor. \\

\subsection{Eclipse Simulator}

The eclipse simulator we delivered to our sponsor and his team has met the
requirements we set forth and is now live on the eclipsemega.movie website. The
eclipse simulator has exceeded our expectations for quality and our sponsor has
expressed his pleasure with our finished product. \\

\subsection{[SUPPLEMENTAL] Eclipse Image Classifier}

The eclipse image classifier is a supplemental piece of the project that our
sponsor asked us to investigate with the remaining time we had on the project.
Currently, we are looking into using Google Cloud Vision (GCV) to classify images
as either partial or total eclipse. This is because we and our sponsor have
decided to add a precondition of the image processor that the images it works
on are of a total eclipse. By using GCV to classify images, we can meet this
precondition. \\

%%%%%%%%%%%%%%%%%%%%%%%%
%   Problems           %
%%%%%%%%%%%%%%%%%%%%%%%%
\section{Problems and Possible Solutions}


%%%%%%%%%%%%%%%%%%%%%%%%
%   Things Left to Do  %
%%%%%%%%%%%%%%%%%%%%%%%%
\section{Things Left to Do}

\subsection{Image Processor}

Lorem \\

\subsection{Image Processor Developer Pipeline}

Lorem \\

\subsection{Eclipse Simulator}

None! \\


%%%%%%%%%%%%%%%%%%%%%%%%
%   Code               %
%%%%%%%%%%%%%%%%%%%%%%%%
\newpage
\section{Interesting Code}

// Code -- image processor?

\begin{minted}{cpp}
int a = b;
\end{minted}

\newpage
\section{Screenshots}

\begin{figure}[!h]
	\begin{center}
  		\includegraphics[width=\textwidth]{sim_total.eps}
		\caption{Simulator in wide mode showing a total solar eclipse}
	\end{center}
\end{figure}
\newpage

\begin{figure}[!h]
	\begin{center}
			\includegraphics[width=\textwidth]{sim.eps}
		\caption{Simulator in wide mode showing no eclipse}
	\end{center}
\end{figure}
\newpage

\begin{figure}[!h]
	\begin{center}
			\includegraphics[width=\textwidth]{sim_map.eps}
		\caption{Simulator with map expanded}
	\end{center}
\end{figure}
\newpage

\begin{figure}[!h]
    \begin{center}
            \includegraphics[width=\textwidth]{imgproc.eps}
        \caption{Image Processor Developer Pipeline Results File}
    \end{center}
\end{figure}
\newpage


%%%%%%%%%%%%%%%%%%%%%%%%
%   Weekly Summary     %
%%%%%%%%%%%%%%%%%%%%%%%%
\newpage
\section{Week by Week Summary of Group Activities}

\subsection{Week 1}

    \begin{itemize}

	\item Lorem

    \end{itemize}

\subsection{Week 2}

    \begin{itemize}

	\item Lorem

    \end{itemize}

\subsection{Week 3}

    \begin{itemize}

	\item Lorem

    \end{itemize}

\subsection{Week 4}

    \begin{itemize}

	\item Lorem

    \end{itemize}

\subsection{Week 5}

    \begin{itemize}

	\item Lorem

    \end{itemize}

\subsection{Week 6}

    \begin{itemize}

	\item Lorem

    \end{itemize}

\newpage
\section{Retrospectives}

\begin{table}[!h]
    \centering
    \begin{tabular}{|p{.3\linewidth}|p{.3\linewidth}|p{.3\linewidth}|}

    \cline{3-3}

    \hline \textbf{Positives} & \textbf{Deltas} & \textbf{Actions} \\ \hline

    Lorem & Lorem & Lorem \\ \hline

    \end{tabular}
\end{table}

\end{document}
