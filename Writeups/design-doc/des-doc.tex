\documentclass[10pt, onecolumn, draftclsnofoot, letterpaper, compsoc]{IEEEtran}

\usepackage{cite}
\usepackage{hyperref}
%\usepackage{enumitem}

% Macro for the signatures
\newcommand*{\SignatureAndDate}[1]{%
    \par\noindent\makebox[2.5in]{\hrulefill} \hfill\makebox[2.0in]{\hrulefill}%
    \par\noindent\makebox[2.5in][l]{#1}      \hfill\makebox[2.0in][l]{Date}%
}

\renewcommand*\contentsname{Table of Contents} % Rename ToC

\newcommand{\myindent}{\hspace{\oldparindent}}

% Temp title and author
\title{Design Document}
\author{Totality AweSun \\
		Bret~Lorimore, Jacob~Fenger, George~Harder \\
		\textit{December 4th, 2016 \\
		CS 461 - Fall 2016}} 

\begin{document}

\maketitle

\begin{abstract}

Lorem ipsum.

\end{abstract}

\vspace{10mm}
\noindent \SignatureAndDate{David Konerding, Project Sponsor}
\vspace{8mm}
\noindent \SignatureAndDate{Bret Lorimore}
\vspace{8mm}
\noindent \SignatureAndDate{George Harder}
\vspace{8mm}
\noindent \SignatureAndDate{Jacob Fenger}

\newpage

\tableofcontents

\newpage

%% Section 1
\section{Design Stakeholders and Their Concerns}

More to come...\cite{OCV}.

\subsection{Image Processor}

\subsubsection{} 

The image process needs to take in an image, identify if the image has a total
solar eclipse, and if it does further process it so that it can be stitched into
a timelapse movie by Eclipse Megamovie engineers. \\

\subsubsection{} 

The mean processing time for an image must be one second. \\

\subsubsection{} 

Images must be processed in no longer than five seconds.\\

\subsubsection{} 

Images must be filtered so that the processed images are only images of the
eclipse at totality.\\

\subsubsection{}

Only high quality images, defined as having a 50 pixel solar disk size and
padding around the disk of 100 pixels, should be accepted by the image
processor.\\

\subsubsection{} 

Once images have been filtered, they need to have metadata attached in a way
that allows easy stitching of eclipse images. \\

\subsubsection{} 

The image processor needs to be able to be called by the image processor manager
with appropriate input data. \\

\subsubsection{} 

Image processor needs to be able to use GPS EXIF information associated with
images.\\

\subsection{Image Processor Manager}

\subsection{Eclipse Simulator}


%% Section 2
\section{Design Viewpoints}

\subsection{Image Processor}

\subsubsection{Speed and Performance}

\textbf{Concerns:} 1.1.2, 1.1.3 \\
\textbf{Elements:} 4.1.1, 4.1.2, 4.1.3, 4.1.4\\
\textbf{Analytical Methods:} The primary criteria and methods in constructing
this view is whether or not we are achieving the desired average speeds on our
golden data set. \\
\textbf{Viewpoint Source:} George Harder \\

\subsubsection{Accuracy}

\textbf{Concerns:} 1.1.1, 1.1.4, 1.1.5, 1.1.8 \\
\textbf{Elements:} 4.1.1, 4.1.2, 4.1.5, 4.1.6 4.1.9, 4.1.10\\
\textbf{Analytical Methods:}  In constructing the corresponding view, we will be
evaluating it based on whether or not the image processor is correctly
identifying eclipses with at least 90\% accuracy on our golden data set.\\
\textbf{Viewpoint Source:} George Harder \\

\subsubsection{Input and Output}

\textbf{Concerns:} 1.1.6, 1.1.7 \\
\textbf{Elements:} 4.1.2, 4.1.4, 4.1.7, 4.1.8, 4.1.9\\
\textbf{Analytical Methods:} This view will be evaluating whether or not the
input and output of the image process meet the specifications defined in the
design document. \\
\textbf{Viewpoint Source:} George Harder\\

\subsection{Image Processor Manager}

\subsection{Eclipse Simulator}


%% Section 3
\section{Design Views}

\subsection{Image Processor}

\subsubsection{Speed and Performance [Governed by Viewpoint 2.1.1]}

The Eclipse Megamovie project expects to receive photographs on the order of
hundreds of thousands from the numerous citizen photographers registered with
the project. This being the case, it is of utmost importance that the image
processor we are building for the project can process images in a timely manner.
This is not only important to being able to build a movie from the images soon
after the eclipse, but also because it prevents storage spaces on either end of
the image processing pipeline from becoming clogged. \\

In order to achieve these design goals, we are designing a system that uses the
OpenCV C++ library.  This API provides us with high performing library methods,
like Hough transforms, image crops, and image rotations that are necessary to
the image processor's core functionality. In addition to the use of this
language and library, we are also designing this system to be single threaded
but to also interface with a manager that runs multiple instances of the image
processor concurrently. This fact allows us to design the image process to
process a single image quickly and accurately and leaves management of the
application to a different art of the system.\\

\subsubsection{Accuracy [Governed by Viewpoint 2.1.2]}

From a high level, the most fundamental concern in the design of the image
processor is that it can accurately identify an eclipse image. The other
concerns associated with the design of the image processor are near meaningless
if the application is not consistently identifying images that contain total
eclipse.\\

To address this concern, and the related concerns around accepting only high
quality images and determining the relative temporal and spatial positioning of
images, we are designing this application with accuracy as a primary goal. The
meet this goal, the system will build upon an existing eclipse identification
algorithm. This algorithm, with improvements we add ourselves, will be the basis
for the parts of the system that we design to meet the accuracy requirements of
the image processor.\\

\subsubsection{Input and Output [Governed by Viewpoint 2.1.3]}

The image processor is one component in a much larger system. For the system to
function the image processor needs to interface with the other components in a
well defined and seamless manner. In addition to speed and accuracy, we need to
design the system with a view toward optimizing the way the image processor
interacts with other components.\\

The ensure a smooth interface with the image processor manager we are designing
the image processor to accept a well defined set of command line arguments
including the path to the list of images to be processed, the path to an output
directory, and the path to prefix image file names with that points to their
location. The design of the image processor also takes into account the needs of
the engineers handling the processed images. The image processor will write
processed images and their metadata in a specific format to an output file.\\

\subsection{Image Processor Manager}

\subsection{Eclipse Simulator}


%% Section 4
\section{Design Elements}

\subsection{Image Processor}

\subsubsection{OpenCV}
\textbf{Type:} Library\\
\textbf{Purpose:} OpenCV is the computer vision library we will use to process
images. This open source library can quickly crop and otherwise manipulate
images. It best suits our needs for a computer vision utility.\\

\subsubsection{C++}
\textbf{Type:} Class\\
\textbf{Purpose:} This application will be written in C++ in order to give us
better control over the speed at which the application runs and the necessary
functionality for reading and writing files. \\

\subsubsection{Image Processing Time}
\textbf{Type:} Constraint\\
\textbf{Purpose:} This element exists because our requirements specify that the
average time for an image to be processed must be less than one second.\\

\subsubsection{Serial Image Processing}
\textbf{Type:} Framework\\
\textbf{Purpose:} The image processor will process images one at a time because
the Image Processor Manager handles all parallelization.\\

\subsubsection{Hough Transform}
\textbf{Type:} Procedure\\
\textbf{Purpose:} The Hough transform is an algorithm for identifying circles
and lines in images. It will be used by the image processor to identify total
eclipse images. OpenCV has the circular Hough transform built in. \\

\subsubsection{Image Quality Error Checking}
\textbf{Type:} Constraint\\
\textbf{Purpose:} The requirements of the image processor specify that only
images with a solar disk of at least 50 pixels and 100 pixels of padding around
the sun will be accepted.\\

\subsubsection{Command Line Arguments}
\textbf{Type:} Constraint\\
\textbf{Purpose:} The image processor needs to have a well defined set of
command line arguments so that it can interface with the image processor
manager without difficulty. \\

\subsubsection{Data writer}
\textbf{Type:} System\\
\textbf{Purpose:} This component is meant to encapsulate the functionality
necessary to write the processed images and their associated metadata to an
output file.\\

\subsubsection{Image Data Structure}
\textbf{Type:} Class\\
\textbf{Purpose:} This class will encapsulate the information and methods needed
to manage the images we will be processing. It will also work closely with the
Data Writer to write images and their metadata to files.\\

\subsubsection{Solar Eclipse Image Standardisation and Sequencing (SEISS)}
\textbf{Type:} Procedure\\
\textbf{Purpose:} The SEISS algorithm is an image processing algorithm that can
identify images of eclipses, including eclipses at totality \cite{imgKrista}. We
will be basing part of the image processor off of this algorithm. \\

\subsection{Image Processor Manager}

\subsection{Eclipse Simulator}


%% Section 5
\section{Design Overlays}

\subsection{Image Processor}

\subsection{Image Processor Manager}

\subsection{Eclipse Simulator}


%% Section 6
\section{Design Rationale}

\subsection{Image Processor}

The design of this system is based on needs surrounding accuracy, speed, and
ability to interface with other components in the larger system. As was detailed
in the technology review, we made certain decisions about what tools and
algorithms to use based on the specific requirements of our system. The use of
these tools has necessarily shaped the design of this application.\\

Additional information regarding design rationale can be found in the `Purpose'\\
section of the design elements.

\subsection{Image Processor Manager}

\subsection{Eclipse Simulator}

%% Section 7
\section{Design Languages}


\newpage

\bibliographystyle{IEEEtran}
\bibliography{des-doc}

\end{document}