\documentclass[10pt, onecolumn, draftclsnofoot, letterpaper, compsoc]{IEEEtran}

\usepackage{cite}
\usepackage{hyperref}
\usepackage{enumitem}

% Macro for the signatures
\newcommand*{\SignatureAndDate}[1]{%
    \par\noindent\makebox[2.5in]{\hrulefill} \hfill\makebox[2.0in]{\hrulefill}%
    \par\noindent\makebox[2.5in][l]{#1}      \hfill\makebox[2.0in][l]{Date}%
}

\renewcommand*\contentsname{Table of Contents} % Rename ToC

\newcommand{\myindent}{\hspace{\oldparindent}}

% Temp title and author
\title{Design Document}
\author{Totality AweSun \\
		Bret~Lorimore, Jacob~Fenger, George~Harder \\
		\textit{December 4th, 2016 \\
		CS 461 - Fall 2016}}

\begin{document}

\maketitle

\begin{abstract}

Lorem ipsum.

\end{abstract}

\vspace{10mm}
\noindent \SignatureAndDate{David Konerding, Project Sponsor}
\vspace{8mm}
\noindent \SignatureAndDate{Bret Lorimore}
\vspace{8mm}
\noindent \SignatureAndDate{George Harder}
\vspace{8mm}
\noindent \SignatureAndDate{Jacob Fenger}

\newpage

\tableofcontents

\newpage

%% Section 1
\section{Design Stakeholders and Their Concerns}

More to come...\cite{OCV}.

\subsection{Image Processor}

\subsection{Image Processor Manager}

\subsection{Eclipse Simulator}
    \subsubsection{}
    The solar eclipse must be accurately simulated based on
    user entered location information.
    \subsubsection{}
    Users will be able to adjust simulator time via a
    draggable slider or clickable buttons.
    \subsubsection{}
    Simulator will only support locations within continental
    United States.
    \subsubsection{}
    The simulator must load in less than 500ms given a 1-10
    Mbps internet connection.

%% Section 2
\section{Design Viewpoints}

\subsection{Image Processor}

\subsection{Image Processor Manager}

\subsection{Eclipse Simulator}
  \subsubsection{Interface}
  \textbf{Concerns:} 1.3.1, 1.3.2 \\
  \textbf{Elements:} 4.3.1, 4.3.2, 4.3.4, 4.3.7 \\
  \textbf{Analytical Methods:} Interface should be appealing
  to the user as well as being responsive and fast. \\
  \textbf{Viewpoint source:} Jacob Fenger

  \subsubsection{Loading Performance}
  \textbf{Concerns:} 1.3.4\\
  \textbf{Elements:} 4.3.3, 4.3.4, 4.3.5, 4.3.6, 4.3.7 \\
  \textbf{Analytical Methods:} The initial loading time
  of the simulator should be fast. Additionally, the
  interactions that the user has with the simulator should
  be responsive and should not show any significant slow
  downs. \\
  \textbf{Viewpoint source:} Jacob Fenger

  \subsubsection{Simulation Accuracy}
  \textbf{Concerns:} 1.3.1, 1.3.3 \\
  \textbf{Elements:} 4.3.3, 4.3.5 \\
  \textbf{Analytical Methods:} In the simulator, the Sun and
  Moon display should reflect scientific accuracy when it
  comes to relative position and sizes. Additionally, the
  view of the Sun and Moon above the horizon shall be
  accurate.\\
  \textbf{Viewpoint source:} Jacob Fenger


%% Section 3
\section{Design Views}

\subsection{Image Processor}

\subsection{Image Processor Manager}

\subsection{Eclipse Simulator}

  \subsubsection{User Interface View [Governed by Viewpoint 2.3.1 ]}
  The user interface shall utilize 2D animated depictions of
  the Sun and the Moon as they appear at a user specified
  time and location. In addition, the user interface will
  contain background imagery such as a city or hillside
  landscape. There will also be a time slider, a location
  input, and a time display for users to interact with or
  view.

  \subsubsection{Operating Performance View [Governed by Viewpoint 2.3.2]}
  The simulator will have low loading times to ensure fast
  performance for most users. Additionally, the simulator
  will need to respond in a timely matter when users are
  interacting with the module.

  \subsubsection{Eclipse Accuracy View [Governed by Viewpoint 2.3.3]}
  The simulator shall be accurate enough for any location in
  the continental United States. This accuracy includes
  accurate relative Moon and Sun sizes, positions in the
  rendered scene, and positions relative to one another.

%% Section 4
\section{Design Elements}

\subsection{Image Processor}

\subsection{Image Processor Manager}

\subsection{Eclipse Simulator}

  \subsubsection{Scalar Vector Graphics (SVG)}
  \textbf{Type:} System \\
  \textbf{Purpose:} This element shall be used for the
  front-end display of the eclipse simulator.
  Two-dimensional images of the Sun and Moon will be altered
  based on how the user interacts with the module.


  \subsubsection{Cascading Style Sheets (CSS)}
  \textbf{Type:} System \\
  \textbf{Purpose:} CSS helps with the front-end display of
  the simulator by helping produce better looking output.

  \subsubsection{Ephemeris JavaScript Library}
  \textbf{Type:} Library \\
  \textbf{Purpose:} This library is used to compute eclipse
  information to be used for displaying the Sun and Moon.

  \subsubsection{View}
  \textbf{Type:} Component \\
  \textbf{Purpose:} Combined the HTML, SVG, and CSS elements
  for simulator display and interaction for the user.

  \subsubsection{Model}
  \textbf{Type:} Component \\
  \textbf{Purpose:} Backend library in JavaScript used for
  computing eclipse information which will be passed to the
  controller. This entity utilized the Ephemeris JavaScript
  library for support.

  \subsubsection{Controller}
  \textbf{Type:} Component \\
  \textbf{Purpose:}  Controls the interaction between the
  model and view. Information will be passed between these
  entities.

  \subsubsection{Model-View-Controller Architecture}
  \textbf{Type:} Relationship \\
  \textbf{Purpose:} This architecture is defined by the interactions of the model, view, and controller entities.

%% Section 5
\section{Design Overlays}

\subsection{Image Processor}

\subsection{Image Processor Manager}

\subsection{Eclipse Simulator}


%% Section 6
\section{Design Rationale}

\subsection{Image Processor}

\subsection{Image Processor Manager}

\subsection{Eclipse Simulator}


The goals for the simulator were to provide a fast and
responsive experience to the user while providing
scientifically accurate results. One main concern with this
is that the time to compute information regarding the Sun
and Moon must be quick enough to not pose any significant
delay for the rest of the simulator.

We decided to utilize the MVC architecture since model and
view components can be exchanged without compromising the
whole system. System designer only need to account how the
components interact with each other to update the system.
In the technology review, we compared two libraries: Suncalc
and Ephemeris. Initially, we chose SunCalc as the better
library due to better documentation and ease of use. After
further testing, the results that Ephemeris was providing
were much better which spurred the change to utilizing
Ephemeris as our support in ephemeris computations. 

Additionally, we chose to utilize a scalar vector graphics
format due to sub-element event processing and being easily
to move the Sun and Moon around as animations.

%% Section 7
\section{Design Languages}


\newpage

\bibliographystyle{IEEEtran}
\bibliography{des-doc}

\end{document}
